\documentclass{memoir}
\usepackage{amsmath}
\usepackage{amssymb}
\usepackage[utf8]{inputenc}
\usepackage{commath}
\usepackage{hyperref}
\usepackage{siunitx}
\usepackage{float}
\usepackage{graphicx}
\usepackage[left=2cm, top=2cm]{geometry}
\usepackage{pgfplotstable}
\usepackage{array}
\usepackage{cleveref}
\usepackage{wrapfig}
\title{Sine integral}
\date{}
\begin{document}
\section*{Sine integral}
\begin{wrapfigure}{r}{0.5\textwidth}
	\input{SinIntPlot.tex}
\end{wrapfigure}
The sine integral is part of the trigonometric integrals. It is defined in the following two ways 
\begin{align*}
	\mathrm{Si}(x)&=\int_{0}^{x}\frac{\sin t}{t}\dif x\\
	\mathrm{si}(x)&=-\int_{x}^{\infty}\frac{\sin t}{t}\dif x
\end{align*}
Both integrals are antiderivatives of the sinc function $\frac{\sin x}{x}$, but pertaining to different limits of the sinc function, as $\mathrm{Si}(x)$ is the antiderivative whose limit is $0$ as $x\rightarrow 0$, while $\mathrm{si}$ is the antiderivative whose value is $0$ as $x\rightarrow\infty$.
The integrals are related to each other by the integral \ref{Dirichlet}, the proof for which can be seen in \nameref{DirProof}
\begin{align}
	\label{Dirichlet}
	\mathrm{Si}(x)-\mathrm{si}(x)&=\int_{0}^{\infty}\frac{\sin t}{t}\dif t=\frac{\pi}{2}
\end{align}
This property is important when implementing the \textit{quad} integration routines, as the infinite integral of the \textit{Sin} function in $C\#$ takes an enormous amount of time.\\
Furthermore, for very large arguments ($x>700$), the solution exhibits sudden noise, uncharacteristic of the actual function. To avoid this, the functions were modified such that $\mathrm{Si}(x>700)=\frac{\pi}{2}$ and $\mathrm{si}(x>700)=0$, as the are already reasonably close to these values at this point.




\appendix
\section*{Appendix A}
\label{DirProof}
Starting with equation \ref{Dirichlet}, it is rewritten into a function of the variable $a$
\begin{align*}
	f(a)&=\int_{0}^{\infty}e^{-at}\frac{\sin t}{t}\dif t
\end{align*}
Where the particular integral used is $f(0)$.
The, utilizing Feynman's trick, we differentiate with respect to $a$
\begin{align*}
	\od{}{a}f&=\od{}{a}\int_{0}^{\infty} e^{-at}\frac{\sin t}{t}\dif t = \int_{0}^{\infty} \pd{}{a}e^{-at}\frac{\sin t}{t}\dif t = -\int_{0}^{\infty} e^{-at}\sin t \dif t
\end{align*}
Then rewriting with $\sin t = -\frac{1}{2i}\big(e^{it}-e^{-it}\big)$
\begin{align*}
	\od{f}{a}&=-\int_{0}^{\infty}e^{-at} \,\frac{e^{it}-e^{-it}}{21}\dif t=\frac{1}{2i}\int_{0}^{\infty}e^{-(a+i)t}-e^{-(a-i)t}\dif t\\
	&= \frac{1}{2i}\Big[ \frac{1}{a-i}e^{-(a-i)t}-\frac{1}{a+i}e^{-(a+i)t}  \Big]_0^{\infty}=-\frac{1}{1+a^2}
\end{align*}
Then applying integration
\begin{align*}
	f=\int \od{f}{a}\dif a=-\int \frac{1}{1+a^2}\dif a=-\arctan a + C
\end{align*}
Determining the integration constant $C$ from limit requirements
\begin{align*}
	\lim_{a\rightarrow \infty}f&=0=\lim_{a\rightarrow \infty}(C-\arctan a)\implies\\
	C&=\lim_{a\rightarrow \infty}\arctan a = \frac{\pi}{2}
\end{align*}
Such that
\begin{align*}
	f(a)= \frac{\pi}{2}-\arctan &a = \int_{0}^{\infty}e^{-at}\frac{\sin t}{t}\dif t\implies\\
	\int_{0}^{\infty}\frac{\sin t}{t}\dif t &= f(0) = \frac{\pi}{2}
\end{align*}



\end{document}
